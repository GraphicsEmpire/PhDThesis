\pagebreak
\newpage
\TOCadd{Abstract}

\noindent \textbf{Supervisory Committee}
\tpbreak
\panel

\begin{center}
\textbf{ABSTRACT}
\end{center}
Modelling deformable tissues has applications in many interesting areas of research including visual effects in the movies, games and
surgical simulation systems. Modelling such objects requires mastering several complex software packages.  Designers often take an incremental 
fine-tuning approach, i.e. a process which is based on many small modifications to the original design. The result of such 
changes are required to be seen in real-time. Creating life-like animations is far more complicated and requires certain skills and experience. 

In a surgical simulation system for example the ability to interact with the models requires accuracy and high performance rendering. 
Cutting and poking the models also impose many challenges. Prototyping models for surgical scenes requires lots of post-processing steps 
in order to bring those models to life.

In this research we propose a comprehensive framework for high-performance rendering and physically-based animation of deformble tissues
using implicit surfaces. Our system provides interactive cutting ability using smooth intersection surfaces. 
Complex models can be created with implicit primitives, blending operators, affine transformations and constructive solid geometry 
in a design environment that organizes all these in a scene graph data structure called BlobTree. We show that the BlobTree modelling approach
provides a very compact data structure which supports incremental changes and network-based cooperative design. 

Using a finite element approach we discetize BlobTree models with a GPU-Assisted algorithm for accurate physically-based animation. Interactions with the
model are supported through smooth cut surfaces and force-feedback probing. Many surgical simulation scenarios are possible to be modelled with 
our proposed framework which are far more complex than what has previously been possible. We show an application of our system in a human skull 
craniotomy simulation where the user is able to drill a hole into the human skull model. 

\pagebreak


