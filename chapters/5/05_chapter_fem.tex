\startchapter{GPU-Accelerated Finite Element Framework}
\label{chapter:finiteelementmethod}
During the past decade a new thrust has appeared with the development of deformable tissue modeling techniques which could make it
possible to develop patient-specific simulations. Whenever the best surgical strategy is unclear or the patient presents a rare
pathology such simulations could be beneficial. Other use-cases for such systems is in surgical skill training which is a long 
and tedius process of acquiring fine motor skills. In all those scenarios physically-based animation of soft tissues is challenging and 
has lots of room for improvement. Based on the foundation developed in the previous chapters here we present a flexible framework for 
real-time soft tissue simulation. In this chaper we present our contribution in modeling and animating soft body deformations using 
our GPU accelerated computational framework and a robust finite element solver. To begin with we review the related works in this
fields and then we present our framework and important design decisions. Finally we present some models animated using our system. 
The materials presented in the following chapters rely on the computational framework that we built here so we strongly suggest
not to skip this chapter.

Surgical simulation systems can benefit from fast computational algorithms that provide visual and haptic feedback to the surgeon. 
Various haptic interfaces for medical simulation are specially useful for training surgeons for minimally invasive procedures e.g.
laparascopy or interventional radiology and remote surgery using tele-operators. The hard requirement for these type of systems is
that the force interaction between the robotic tool and the tissue should be computed as speeds of at least 500 Hz to provide the 
continuous force-feedback to the surgeon \cite{dimaio2005interactive, Otaduy2005}.

There are difference strategies to reduce the computation time:

\begin{enumerate}
 \item Improving the algorithms involved in soft tissue simulation (new force models)
 \item Leveraging faster hardware (Higher computational powe)
 \item Using parallel computing techniques (Multicore and Manycore optimizations)
\end{enumerate}

We considered all these options but the use of faster hardware in our case is limited by the existing technology but appropriate 
use of parallel computing techniques helped us in providing scalable solution at the expense of more expensive hardware and more 
complex implementations.


In designing a finite element solution there are many aspects that must be considered such as the type of formulation used e.g.
Total or updated Lagrangian, time integration scheme e.g. explicit or implicit and the type of elements used for discretization:
e.g. hexahedral (uniform cubic voxel grid or non-uniform octree based grid) or n-simplices such as tetrahedral elements. 
These design choices will impact the performance and the range of applications that can benefit from this formulation.
We will discuss these aspects in greater detail in this chapter. 


\section{Related Work}
In their survey paper Meier \etal \cite{Meier2005} presented a classification of the deformable models available for surgery 
simulation based on their applications. The main classifications are heuristic, hybric and continuum mechanical approaches.
Examples of heuristic models are: deformable splines, spring-mass models, linked volumes and tensor-mass models. 

With deformable splines, classical splines are employed to model a 3D object. They define a potential energy which is proportional
to the degree of elastic deformation. By using the lagrange method, this energy is finally minimized with respect to the displacements
enforced in some control points to obtain the corresponding deformation state. The increased number of parameters required to control
the shape and the physical properties of the models will become the bottleneck and are difficult to determine empirically. A post processing
step is also required to convert the smooth spline surface to discretized polygons for rendering. Even without this post processing stage
they are computationally very expensive. E.g. solid objects have to be modelled as hollow shells which is not lending itself very well for
volume preservation applications.

\subsection{Mass-Spring Models}
Mass spring models (MSM) are the most intuitive and most common method used in deformable object modeling.
Terzopoulos \etal \cite{terzopoulos1988modeling, terzopoulos1987elastically} was the first who introduced mass-spring systems in 
computer graphics animation systems. Provot \etal showed applications of this technique in cloth simulation \cite{provot1995deformation}
as did Baraff \etal \cite{baraff1998large}. Kahel \etal used in facial animation \cite{kahler2001geometry}. 
Gibson \etal survey reviewed many examples of mass-spring system applications \cite{Gibson1997a}. In this system the deformable object
is discretized into a network of point masses which are connected with massless springs and dampers. By adding weights and considering
constant coefficient $k_s$ the elastic force is obtained by:

\begin{equation}
 \boldsymbol{f}^s_i = \sum_{j \in N(\boldsymbol{x}_i)} k_s w_{ij}(\boldsymbol{x}_{ij})(1-\frac{l^0_{ij}}{\| \boldsymbol{x}_{ij} \|})
\end{equation}

In this setup:

\begin{itemize}
 \item $\boldsymbol{x_{ij}} = \boldsymbol{x}_j - \boldsymbol{x}_i$ where $\boldsymbol{x}_i$ is the position of node $i$, 
 \item $l^0_{ij}$ is the initial spring length between node $i$ and node $j$,
 \item $N(\boldsymbol{x_i})$ is the set of neighbors of node $i$ and $w_{ij}$ is the weight of $j$th neighbor of node $i$.
\end{itemize}

The damping force in this setup is given by:

\begin{equation}
 \boldsymbol{f}^d_i = \sum_{j \in N(\boldsymbol{x}_i)} k_d w_{ij} \frac{\boldsymbol{v}^\tau_{ij} \boldsymbol{x}_{ij} }{\|\boldsymbol{x}_{ij} \|} \boldsymbol{x}_{ij}
\end{equation}
where $\boldsymbol{v}_i$ is the velocity of node $i$ and $\boldsymbol{v}_{ij} = \boldsymbol{v}_j - \boldsymbol{v}_i$.

\subsection{Linked Volumes}
The ideas in MSM can be extended to volumetric modeling by discretizing the entire volume of a deformable model into evenly spaced cubic elements.
The total mass of the model is distributed at the centers of these cubes. The cubes are interconnected with their neighbors using springs and dampers
\cite{gibson1997simulating}. One of the disadvantages of this approach is its increased computational cost due to the higher number of nodes and
their increased connectivity. In order to control the computational cost for large models the resolution of the discretization should be lowered 
which will lead to unrealistic deformations. For rendering purposes auxilary surface representations should be used such as surface maps. 






























