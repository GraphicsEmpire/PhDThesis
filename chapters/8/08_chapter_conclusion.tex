\startchapter{Conclusions}
\label{chapter:conclusion}
We presented a comprehensive system for modelling and animating soft-tissues based on implicit surfaces and volumetic range data-sets.
Our system provides the following facilities for researchers in the field of surgical simulation:

\begin{itemize}
 \item A comprehensive modeling framework supporting a broad set of skeletal implicit primitives, sketched primitive objects, warping, blending, 
 affine tranformations and constructive solid geometry operators in the compact \blob structure.
 \item An algorithm for interactive polygonization of implicit surfaces on multi-core architectures with SIMD instructions.
 \item An optimized GPU-assisted algorithm for high-performance polygonization of implicit surfaces on many-core architectures.
 \item A high-performance algorithm for volume discretization of \blob models which can generate tetrahedral mesh elements with respect to a triangular 
 surface mesh for finite element formulation.
 \item A non-linear finite element formulation of deformations and a GPU-assisted algorithm for fast numerical approximations.
 \item A high-performance algorithm for cutting tissues interactively. 
\end{itemize}

The results presented in the previous chapter showcases only a small set of possible simulation scenarios. We believe that our 
system can aid in other complex surgical tasks such as Hysterectomy or Hepatectomy as well. 

There are many areas in which the proposed system may be improved. The polygonization method introduced in chapter \ref{chapter:GPUDiscretization} can be further 
extended to support for more complex implicit primitives such as skeletal curve primitives. Such primitives can be helpful in modelling vains and other tube-shaped
tissues. Support for implicit decals as suggested in \cite{Schmidtb} can help in creating realistically textured organs. 

Our cutting algorithm can also be extended to support progressive cuts. Progressive cuts can drastically enhance the perceived sense cutting. 
Also incorporating a fluid simulation will enhance the cutting scenario for simulating blood and CSF fluids in the brain.
Procedural models can be used to simulate the small blood vessels in the brain biopsy simulation. 

The gains that have been made during the course of this research, have thus only begun to be explored. As computational
power continues to increase, and optimization algorithms continue to improve, implicit surfaces are likely to play a much larger role in 
computer graphics. It is the sincere hope of the author that this research demonstrates what that role might be, and will encourage others
to explore this domain.
