\startchapter{Conclusions}
\label{chapter:conclusion}
The following contributions were made:


\begin{enumerate}
  
\item A comprehensive modelling framework supporting a broad set of skeletal implicit primitives, 
 sketched primitive objects, warping, blending, affine transformations and constructive solid geometry 
 operators in the compact \blob structure. Our framework also provides a software architecture for 
 physically-based animation of rigid and deformable models.
 
 \item An algorithm for interactive polygonization of implicit surfaces on multi-core architectures with 
 SIMD instructions (peer reviewed). 
 
 \item An optimized GPU-accelerated algorithm for high-performance polygonization of implicit surfaces 
 on many-core architectures. 
 
 \item A novel mesh data-structure suitable for storing dynamic meshes on the GPU to support realtime 
 modifications during cutting
 
 \item Smooth, interactive cutting for complex elastic and rigid tissues 
 
% \item A novel technique for collision detection using implicit fields which is used in our system to detect 
%the intersection of the scalpel tool with the volume mesh in real-time. 
 
\item A real-time Craniotomy simulation for neurosurgery and biopsy simulations.

\end{enumerate} 

% 1. modelling framework
Our modelling framework enables physically-based animation of deformable models. 
This is better than what has been done by Cani \etal \cite{Grascuel1997}.  

% 2. SIMD polygonization method
Our SIMD polygonization method is peer reviewed \cite{Shirazian2012}. 
The proposed algorithm is scalable, dynamic and data-driven as opposed to the related work in this 
area where the input model can be either simple static functions or constant range data-sets 
\cite{Johansson2006, Tatarchuk2007, Knoll2007, Yang2010}.

% 3. GPU accelerated polygonization
Our proposed SIMD polygonization method is later optimized for using GPU acceleration. The 
proposed compact data-structure for \blob in section \ref{sec:datastructure} enables the transfer and 
rendering of large \blob in the order of 60,000 nodes interactively (as shown in that chapter using our 
compact data-structure a 64K nodes \blob only takes about 20 MB in video memory).  The result is a 
high performance polygonization method that enabled real-time updates in our incremental modelling 
system. This result is better than the work of \cite{Knoll2007, Yang2010, singh2010real, chochlik2012gpu}.
 
% 4. GPU based dynamic mesh
An intuitive volumetric mesh data-structure is proposed which is suitable for storing dynamic meshes on 
the GPU to support realtime modifications during cutting. Our cutting results show that the presented 
data-structure is more performant and can benefit the related work in this domain 
\cite{Wu2004a, Wu2005, Courtecuisse2010}.

% 5. cutting contribution how does it compare to previous work
Our proposed GPU-based data-structure enables real-time updates of the volumetric meshes upon 
cutting. Our cutting algorithm as shown in section \ref{sec:cutalg} is interactive for approximately 100,000 finite 
element cells. This is sufficient for highly complex brain surgery simulations. The resulting finite element 
cells are of high quality (the tetrahedral cells are not flat or wedge-shaped) as shown in section \ref{sec:cutres}. 
This is better than what has been done in the work of Courtecuisse \etal \cite{Courtecuisse2010}.
The cut edges are smooth, not jagged and a minimal amount of tetrahedral elements are 
created as the result of elements subdivision and this is better than the results published by Courtecuisse
\etal and Steinemann \etal \cite{Courtecuisse2010, Steinemann}. 


% 7. Craniotomy simulation
We presented a Craniotomy simulation based on our real-time cutting algorithm and the segmented 
brain data-set published by Fang \etal \cite{fang2010mesh}. 
%Although at this point we don't support haptic feedback but the simulation is well-received by one neurosurgeon at Stanford school of 
%medicine and Dr. Sandrine deRibaupierre from Western University. 
Given the large number of finite element cells in the brain mesh (around 100,000), our 
simulation still runs at interactive rates and the cutting output is smooth for a high-quality simulation 
as shown in section \ref{sec:craniotomy}. Colchester \etal used superimposed surface mesh for 
guiding a Craniotomy simulation \cite{Colchester1994}. Abe \etal used plastic skull models for 
training this procedure \cite{Abe1998}. To the best of our knowledge our proposed method is the only 
physically-based simulation for this specific procedure. 




\section{Future Work}
There are many areas in which the proposed system may be improved. The polygonization method 
introduced in chapter \ref{chapter:GPUDiscretization} can be further extended to support for more 
complex implicit primitives such as skeletal curve primitives. Such primitives can be helpful in modelling 
vain and other tube-shaped tissues. Support for implicit decals as suggested in \cite{DeGroot2014} can help 
in creating realistically textured organs. 

Our cutting algorithm can also be extended to support progressive cuts. Progressive cuts can drastically 
enhance the perceived sense cutting. Also incorporating a fluid simulation will enhance the cutting 
scenario for simulating blood and CSF fluids in the brain. Procedural models such as L-Systems can be 
used to simulate the small blood vessels in the brain biopsy simulation. 

As computational power continues to increase, and optimization algorithms continue to improve, implicit 
surfaces are likely to play a much larger role in computer graphics. It is the sincere hope of the author 
that this research demonstrates what that role might be, and will encourage others to explore this domain.



