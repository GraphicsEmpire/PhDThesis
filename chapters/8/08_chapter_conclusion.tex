\startchapter{Conclusions}
\label{chapter:conclusion}
Surgical simulation systems can benefit from fast computational algorithms that provide visual and haptic feedback to the surgeon. 
Various haptic interfaces for medical simulation are specially useful for training surgeons for minimally invasive procedures e.g.
laparascopy or interventional radiology and remote surgery using tele-operators. The hard requirement for these type of systems is
that the force interaction between the robotic tool and the tissue should be computed as speeds of at least 500 Hz to provide the 
continuous force-feedback to the surgeon \cite{dimaio2005interactive, Otaduy2005}.

One can follow various strategies to reduce the computational time:

\begin{enumerate}
 \item Improving the algorithms involved in soft tissue simulation (new force models)
 \item Leveraging faster hardware (Higher computational power)
 \item Using parallel computing techniques (Multicore and Manycore optimizations)
\end{enumerate}

We considered all these options but the use of faster hardware in our case is limited by the existing technology but appropriate 
use of parallel computing techniques helped us in providing scalable solution at the expense of more expensive hardware and more 
complex implementations.

Towards this end, we presented a comprehensive system for modelling, interaction and rendering soft-tissues based on implicit surfaces and volumetric range data-sets.
Our system provides the following facilities for researchers in the field of surgical simulation:

\begin{itemize}
 \item A comprehensive modelling framework supporting a broad set of skeletal implicit primitives, sketched primitive objects, warping, blending, 
 affine transformations and constructive solid geometry operators in the compact \blob structure.
 \item An algorithm for interactive polygonization of implicit surfaces on multi-core architectures with SIMD instructions.
 \item An optimized GPU-assisted algorithm for high-performance polygonization of implicit surfaces on many-core architectures.
 \item A high-performance algorithm for volume discretization of \blob models which can generate tetrahedral mesh elements with respect to a triangular 
 surface mesh for finite element formulation.
 \item A non-linear finite element formulation of deformations and a GPU-assisted algorithm for fast numerical approximations.
 \item A high-performance algorithm for cutting tissues interactively. 
\end{itemize}

There are many areas in which the proposed system may be improved. The polygonization method introduced in chapter \ref{chapter:GPUDiscretization} can be further 
extended to support for more complex implicit primitives such as skeletal curve primitives. Such primitives can be helpful in modelling vain and other tube-shaped
tissues. Support for implicit decals as suggested in \cite{Schmidtb} can help in creating realistically textured organs. 

Our cutting algorithm can also be extended to support progressive cuts. Progressive cuts can drastically enhance the perceived sense cutting. 
Also incorporating a fluid simulation will enhance the cutting scenario for simulating blood and CSF fluids in the brain.
Procedural models can be used to simulate the small blood vessels in the brain biopsy simulation. 

As computational power continues to increase, and optimization algorithms continue to improve, implicit surfaces are likely to play a much larger 
role in computer graphics. It is the sincere hope of the author that this research demonstrates what that role might be, and will encourage others
to explore this domain.
