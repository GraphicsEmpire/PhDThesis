%\newcommand{\etal}{et al. }
%\newcommand{\blob}{\textit{BlobTree }}
%\newcommand{\blobns}{\textit{BlobTree}}


\startfirstchapter{Introduction}
\label{chapter:introduction}
\section{General aim of this Research}
The goal of this thesis is to develop a framework for physically based animation of soft tissues which are increasingly used in
games, movies and simulation systems. Although at this stage this physically-based model is a goal in itself, it is hoped that
it will eventually be used for clinical predictions and for research on the influence of deformations on the functional properties 
of deformable tissues. In order to demonstrate this potential, preliminary studies in that direction are reported in the final chapter
of this thesis. To that end, a skull craniotomy simulation scenerio is created which showcases the contributions of this research in an
interactive surgical scenario. 

In this research we tackled three main problems found in the domain of soft tissue simulation and has been the topic of interest for many
researchers in the field. First, our proposed modeling solution captures the key advantages found in volumetric modeling approaches using 
implicit surfaces \cite{Bloomenthal1997, Wyvill1986, Wyvill1999, Wyvill1996, Wyvill1997, Schmidt2006, Bernhardt2010a}. Automatic blending and compact 
representation are the major benefits of using implicit surfaces for modeling. In addition, the ability to perform inside-outside tests 
easily is an inherent advantage in implicit models when implementing physically based simulations requiring collision tests. 
The \blob \cite{Wyvill1999} combined blending, affine transformations and constructive solid geometry (CSG) operators in a 
comprehensive and compact scene graph data-structure. \blob provides the ability to create complex models incrementally \cite{Schmidt2006}. 
However, volumetric models in general are often several orders of magnitude slower during visualization \cite{Bloomenthal1990a, Bloomenthal1997}.
We proposed a data-driven algorithm for rendering complex implicit models in realtime on multi-core processors \cite{Shirazian2012}, later, we 
fine tuned that algorithm for running on many core architectures such as the ones in high-end graphical processing units (GPUs). 

In traditional animation systems based on key-framing, specifying the motion and the successive shapes of objects interacting with a simulated 
world requires a great amount of specialized knowledge and intuition from the animator. Models based on simplified physical laws have been 
proposed for automating these tasks. They generate motion and deformation from initial conditions and from a set of externally applied forces 
over time, and automatically detect and respond to collisions. These models are particularly appropriate for facilitating the animation of 
deformable objects. They can either be used alone for the simulation of inanimate bodies, or be combined with user-controlled structures as 
has been done for instance in character animation \cite{chadwick1989layered,miller1988motion}. We show that a number of problems that are difficult
to solve with previous models can be easily handled by combining them with an external layer based on implicit surfaces. The implicit formulation
defines a smooth surface around the object that can be used to perform efficient collision detection, to enable exact contact modeling, and to ease
volume preservation. The second contribution of our research is the high performance volume discretization technique and non-linear finite element 
formulation to support the elastic deformations of the created models \cite{Shirazian2013}. 

In modern interactive simulation and modeling enviroments the ability to cut 3-dimensional geometry in realtime is of fundamental importance. 
This creates the need for efficient cutting algorithms that process the underlying representation. Such methods can be utilized in a wide spectrum 
of applications including surgical interventions, free form modeling, or scientific visualization. In surgery simulation, for instance, interactive 
cutting algorithms enable the dynamic simulation of scalpel intersections that open immediately behind the scalpel \cite{Nienhuys2001}. In the 
case of free-form modeling or sculpting, dynamic cutting supports a precise positioning and guidance of a cutting tool. In scientific visualization, 
real-time cutting algorithms create new opportunities for the interactive analysis of volume data sets. Seismic data sets, as an example, can be 
cut arbitrarily along interesting strata. The third main contribution of our research is our GPU-assisted interactive cutting algorithm that allows
arbitrary cuts in the model and can enable many scenarios for tissue manipulation and sensing. 

In what follows, the implicit modeling approach to deformable tissue modeling will be studied. To achieve the initial goal of this research, a
computational framework for designing, rendering and animating deformable tissues has been developed which has the potential to be used in any 
simulation system requiring deformable tissues with some level of interaction and topolgy modification. The final chapter of this thesis is dedicated
for the evaluation of our system and analysis of some example scenarios in which our system might be a good fit for. 


\section{Deformable Models}
Deformable models can be defined in either one dimension (lines and curves), two dimensions (surfaces), or three dimensions (solid objects). 
Essentially, they are applied in three different areas of research \cite{Meier2005}: 

\begin{itemize}
 \item Object modeling for pre-computed animations \cite{coquillart1990extended, hsu1992direct}.
 \item Image segmentation (automatic 2D interpretation of the images provided by a camera or 3D reconstruction of organs from medical MRI 
 or CT scans) \cite{neveu1994recovery}.
 \item Interactive medical simulations i.e. to emulate the deformational behaviour of non-rigid objects due to external influences.
\end{itemize}

Our solution is targeted for the last area where it can be used both in deferred application like surgery planning (e.g., simulation of the 
outcome of craniofacial surgery) \cite{bro1995modelling, keeve1996craniofacial} and in real-time applications. These real time applications 
include image guided surgery \cite{Szekely2000}, minimally-invasive or tele-surgery and surgery simulation. 

From the number of applications, it is obvious that there is no single deformable model that is appropriate for all of the above mentioned 
problems. Instead, there are a variety of methods that are optimized in different ways to meet specific needs. Even though virtual reality 
applications of deformable models are becoming more and more frequent, in many cases, the governing prerequisite for the simulation of 
mechanical deformations has been interactivity rather than precision. In addition to that, many of the modelled objects like garments or soft 
tissues do not possess easily describable properties. Consequently, exact methods, in general, cannot be applied as deformable models, and thus, 
other approaches must to be found \cite{bro1998finite}. Important progress in the field of deformable models has been made since the emergence of surgery 
simulation, with one of the first contributors being Cover et al. \cite{cover1993interactively}. This is mostly due to the extreme prerequisites 
as far as computation time, complex properties of the simulated soft tissues, and intricate interactions with the virtual instruments are concerned. 
In fact, there have been many different points of departure in the research of adequate deformable models, focusing on anatomies and surgical techniques 
that are as different from each other as are eye surgery \cite{cai2001parametric, sagar1994virtual}, knee arthroscopy \cite{gibson1997simulating, 
hoffman1998commercially}, or hepatic laparoscopy \cite{cotin1999real}. The deformable models developed in this context can be divided into three basic groups: 

\begin{enumerate}
 \item The ad-hoc heuristic methods
 \item Simplified continuum-mechanical model
 \item Hybrid methods from the combination of 1 and 2
\end{enumerate}

\section{Motivation}
One of the most important changes in the past decade has been the laparoscopic surgery which brought new technologies into the operating room and created a 
distance between the surgeon and the patient. More recently, other minimally invasive techniques have been proposed, such as natural orifice transluminal 
endoscopic surgery, which can be considered as an evolution of the laparoscopic surgery. As a new surgical technique, laparoscopy requires surgeons to acquire 
new skills, and adapt to changes from conventional open surgery (e.g. amplified tremor, diminished tactile sensation, loss of depth perception). This has been 
a motivation for a number of works in the field of surgery simulation, real-time deformable models, or haptic rendering \cite{Lin2004}. The following benefits are reported
from using the surgical simulation systems:

\begin{itemize}
 \item Systematic training and objective assessment of technical competence
 \item Skills learned thanks to the simulator are transferrable to the operating room
 \item The ability to create patient-specific simulations (i.e. rare pathological cases or when the best surgical strategy is unclear.)
 \item The ability to use augmented reality for image-guided surgery (i.e. to improve the accuracy and limit the adverse effects of surgery)
\end{itemize}

In order to accomplish these goals, accurate, real-time biomechanical models are needed, but their interactions with medical devices also needs to be modeled.
Such interactions not only involve tissue manipulation, but also tissue dissection. 

In this context, modeling and high-performance rendering of soft-tissues are the core requirements for any simulation scenarios. 
The development of fast algorithms to compute the deformation, contact response, cutting and haptic feedback of soft tissues could enable a number of the 
aforementioned applications.

More specifically, when considering requirements for realistic interactive simulations of medical procedures, several elements seem mandatory: anatomical 
models and tissue properties need to be patient-specific and obtained without complex additional procedure; soft tissue behaviour needs to be realistic and 
demonstrate a predictive capability, yet it should be compatible with real-time computation; interactions with the surrounding anatomy and with medical 
devices need to involve advanced contact models that can be computed in real-time; the different types of dissection performed on soft tissues should be 
simulated; and finally realistic visual and haptic feedback should be provided to create a higher level of immersion, in particular during training sessions.

\section{Limitation of Current Models}
%%Chapter \ref{chapter:background} will provide more details on the state of the art techniques in soft tissue modelling. 
Among the numerous publications in the field of biomechanics, realtime deformable models, collision detection, contact modelling or haptics, 
few methods have been proposed to address at least a majority of the requirements listed above. Among the existing approaches which at least 
partially aim at this objective, we can cite methods based on spring-mass networks, methods based on linear elasticity, and explicit finite element models 
for non-linear materials \cite{Gibson1997a,Meier2005}. In chapter \ref{chapter:background} we discuss these methods in detail.  

Mass-spring networks are quite simple to implement and very fast to compute, but they fail to properly characterize soft tissues deformation as they 
introduce artificial anisotropy through the choice of the mesh, and make it difficult to relate spring stiffness to material properties such as Young modulus 
\cite{Courtecuisse2010}.

Most methods based on the Linear elasticity made the assumption of small displacements and relied on pre-computed response in order to accelerate the 
computations. The small strain assumption is very restrictive. In addition, during any topogical modifications e.g. in cutting, the pre-computed values 
has to be recalculated which masks their effectiveness in the overall performance of the system.

Cutting deformable tissues is one of the most sought after features in a surgical simulator. In interactive system with high expectations of realism and 
performance the implementation of topogical modifications can become very complex. The proposed solutions suffer from smoothness of the cutting plane or slower
solve time due to lots of extra nodes added to the system. In chapter \ref{chapter:Cutting} we review all the related work in this topic and present our
high performance cutting algorith which is built into our physically-based simulation system.


\section{Contributions}
Contributions described in this thesis fall into four broad categories: a modeling system to create complex deformable tissues under the heading of the 
\blob; a high-performance subsystem for rendering; a non-linear finite element formulation of the deformable models created with \blob scenegraph and
a high-performance topology modification algorithm to support cutting. 

The main contributions are as the following:

\begin{itemize}
 \item A comprehensive modeling framework supporting a broad set of skeletal implicit primitives, sketched primitive objects, warping, blending, 
 affine tranformations and constructive solid geometry operators in the compact \blob structure.
 \item An algorithm for interactive polygonization of implicit surfaces on multi-core architectures with SIMD instructions.
 \item An optimized GPU-assisted algorithm for high-performance polygonization of implicit surfaces on many-core architectures.
 \item A high-performance algorithm for volume discretization of \blob models which can generate tetrahedral mesh elements with respect to a triangular 
 surface mesh for finite element formulation.
 \item A non-linear finite element formulation of deformations and a GPU-assisted algorithm for fast numerical approximations.
 \item A high-performance algorithm for cutting the tissues interactively. 
\end{itemize}


\section{Overview}
In the next chapter we start by providing background material on implicit modelling technique, the \blob scenegraph and the concept of sketch-based, incremental modelling.
We continue by reviewing the physics properties of the deformable tissues and cover some topics on continuum mechanics concepts and force models used in
our system to achieve non-linear deformations. Chapter \ref{chapter:cpuPoly} presents our rendering framework to visualize complex \blob models using multi-core 
architectures. Building on the outcomes of chapter \ref{chapter:cpuPoly}, the improved results are reviewed in chapter \ref{chapter:GPUDiscretization}. 
The discretization technique to convert a \blob model to a physical system is also given in this chapter. Chapter \ref{chapter:finiteelementmethod} is 
where we discuss our accelerated finite element implementation on the GPU side. Previous attempts in modeling deformable tissues are reviewed in this chapter and
the benefits and drawbacked of all the methods are discussed. 

Chapter \ref{chapter:Cutting} presents one of the main contributions of this thesis which is the high performance soft tissue cutting. After a brief overview of the 
related work we present our novel technique in cutting complex soft tissues interactively. Chapter \ref{chapter:evaluation} showcases a skull craniotomy simulation 
scenario and provides comments on the operation itself and the achieved results.

Chapter \ref{chapter:conclusion} provides a summary of the results in the previous chapters and reviews the limitations of the current system
and some discussions on the future work in this research topic.

















