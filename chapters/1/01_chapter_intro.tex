%\newcommand{\etal}{et al. }
%\newcommand{\blob}{\textit{BlobTree }}
%\newcommand{\blobns}{\textit{BlobTree}}


\startfirstchapter{Introduction}
\label{chapter:introduction}
\section{General aim of this Research}
Simulating the behavior of elastic objects in real time and with some level of user interaction is a challenging problem 
with many applications in different areas of research including virtual surgery \cite{Meier2005}. 
A complete model should be quite realistic, interactive and should enable the user to modify the topology of the objects.
Rendering such models on commodity hardware and supporting high fidelity which in general implies high 
accuracy and interactive frame rates are two contrasting constrains. This is due 
to the lack of algorithms and techniques that can efficiently map the modelling problems in this domain to the 
parallel architecture of the graphics processor to support high quality generation and rendering of those objects. 

A number of proposals have been recently presented to fulfill these objectives. But even if the efficiency
of the simulation models have been largely improved in the last few years, object modelling and deformation remains a rather 
complex task and can be solved in interactive time only on models composed by a few hundreds of cells. In addition, integrating effects
such as cut and lacerations, makes the simulation model more complex. In modern interactive simulation environments the ability to 
cut 3-dimensional geometry in real-time is of fundamental importance. This creates the need for efficient cutting algorithms that process the 
underlying representation. In surgery simulation, interactive cutting algorithms enable the dynamic simulation of scalpel intersections that open 
immediately behind the scalpel \cite{Nienhuys2001a}. Cutting a volumetric mesh under deformation is a non-trivial problem, due to several conflicting
requirements. On the one hand, the cutting process should not create badly shaped elements, which could cause numerical instabilities during deformation
calculation. On the other hand cut trajectory should be closely approximated for realistic appearance. So far, most methods have concentrated only on 
one of these problems. 

Therefore the following major problems are identified in the surgical simulation domain:

\begin{itemize}
 \item Modelling complex tissues that are readily available for simulation \cite{Nealen2006,Meier2005,Gibson1997a}.
 \item Real-time visualization of those tissues \cite{Mario2010PolygonMesh,Bloomenthal1997}.
 \item Performing interactive mesh connectivity modifications on complex models 
 while under deformation \cite{Jin2013, Wu2011, Courtecuisse2010, Jerabkova2010}. 
\end{itemize}

We present a comprehensive solution to these problems as following. 
First, our proposed modelling solution captures the key advantages found in volumetric modelling approaches using 
implicit surfaces \cite{Bloomenthal1997, Wyvill1986, Wyvill1999, Wyvill1996, Wyvill1997, Schmidt2006, Bernhardt2010}. Automatic blending and compact 
representation are the major benefits of using implicit surfaces for modelling. In addition, the ability to perform inside-outside tests 
easily is an inherent advantage in implicit models when implementing physically based simulations requiring collision tests. 
The \blob \cite{Wyvill1999} combined blending, affine transformations and constructive solid geometry (CSG) operators in a 
comprehensive and compact scene graph data-structure. \blob provides the ability to create complex models incrementally \cite{Schmidt2006}. 

Secondly we propose a solution for high performance and scalable visualization of complex models created by \blob method.
Volumetric models in general are often several orders of magnitude slower during visualization \cite{Bloomenthal1990a, Bloomenthal1997}.
We propose a data-driven algorithm for rendering complex implicit models in real-time on multi-core processors \cite{Shirazian2012}, later, we 
fine tune that algorithm for running on many core architectures such as the ones in high-end graphical processing units (GPUs). 

Third, we take a different approach to develop a stable and realistic cutting system. Our GPU-assisted interactive cutting algorithm allows
arbitrary cuts in the model and can enable many scenarios for tissue manipulation while under deformations. 

In what follows, the implicit modelling approach to model design will be studied. To achieve the initial goal of this research, a
computational framework for designing, rendering and animating tissues has been developed and the details of the process is documented 
in the following chapters. 


\section{Motivation}

%%these bits should go under requirements
Laparoscopic surgery brought new technologies into the operating room and created a 
distance between the surgeon and the patient. More recently, other minimally invasive techniques have 
been proposed, such as natural orifice transluminal endoscopic surgery, which can be considered as an 
evolution of laparoscopic surgery. The new surgical techniques developed in Laparoscopy
required the surgoens in the field to go under training to adapt themselves to the new environment.
With the loss in depth perception and decreased levels of tactile sensation, the new technique was
certainly different from conventional open surgery.

Without open organ surgery, modern surgeons do not get training in the motor skills of the 
previous generation. Thus there is a motivation to develop realistic surgical simulations using 
real-time deformable models, and haptic rendering for further realism \cite{Lin2004}. 
The following benefits have been reported from using surgical simulation systems 
\cite{Liu2003, Basdogan2004}: 

\begin{itemize}
 \item Systematic training and objective assessment of technical competence
 \item Skills learned thanks to the simulator are transferable to the operating room
 \item The ability to create patient-specific simulations (i.e. rare pathological cases or when the best 
 surgical strategy is unclear.)
 \item The ability to use augmented reality for image-guided surgery (i.e. to improve the accuracy and 
 limit the adverse effects of surgery)
\end{itemize}

%These are the benfits
In order to achieve these benefits, accurate, real-time bio-mechanical models are needed together
with interactions with medical devices. Such interactions involve tissue manipulation and tissue dissection. 

In this context, modelling and high-performance rendering of tissues are the core requirements for 
a simulation scenario. The development of fast algorithms for rendering, contact response, 
cutting and haptic feedback of soft tissues can enable a number of the aforementioned applications.


\section{Limitation of Current Models}
\label{sec:limitationsOfCurrentModels}
%%Requirements listed below and then 

The requirements listed below are considered in our design of a robust modelling system for simulation:

\begin{itemize}
  \item Complex models should be created incrementally from simpler, reusable primary components
  \item Rendering should be interactive and the result of modifications to the model should be
  seen in real-time.
  \item Interactions with the surrounding anatomy and with medical devices need to involve advanced 
  contact models that can be computed in real-time.
  \item Guided cutting with scalpel and other variants of dissection operations should be supported.
   
  \item Modelling and simulation should be tightly coupled to enable the design of realistic environments 
  for surgical training.
\end{itemize}

%%Chapter \ref{chapter:background} will provide more details on the state of the art techniques in soft tissue modelling. 

%note to self
Popular existing techniques are based on mass-spring networks, methods based on linear elasticity, and explicit 
finite element models for non-linear materials \cite{Gibson1997a,Meier2005}. 

Mass-spring networks are quite simple to implement and very fast to compute, but they fail to properly 
characterize soft tissues deformation as they introduce artificial anisotropy through the choice of the 
mesh, and make it difficult to relate spring stiffness to material properties such as Young's modulus 
\cite{Courtecuisse2010}.

Most methods assume Linear elasticity based on small displacements and pre-computed 
response values to accelerate the computations. The small strain assumption is very restrictive. In 
addition, during any topological modifications e.g. in cutting, the pre-computed values 
have to be recalculated which masks their effectiveness in the overall performance of the system.

Cutting deformable tissues is one of the most sought after features in a surgical simulator. In an interactive 
system with high expectations of realism and performance, the implementation of topological 
modifications can become very complex. The proposed solutions suffer from smoothness of the cutting 
plane or slower solve time due to lots of extra nodes added to the system. In chapter 
\ref{chapter:Cutting} we review all the related work in this topic and present our high performance 
cutting algorithm which is built into our physically-based simulation system.


\section{Contributions}
\label{sec:contributions}
In this research we take into account the requirements of modern surgical procedures and introduce a 
new modelling framework with the ability to perform real-time topological changes (cutting) in a 
non-linear model, providing more realistic behavior than that can be provided by existing models. 

Contributions described in this thesis fall into four broad categories: a modelling system to create 
complex deformable tissues under the heading of the \blob; a high-performance subsystem for 
rendering; a high-performance topology modification algorithm to support cutting and a  
real-time Craniotomy simulation for neurosurgery simulation. The full contributions list is as 
following:

\begin{itemize}
 \item A comprehensive modelling framework supporting a broad set of skeletal implicit primitives, 
 sketched primitive objects, warping, blending, affine transformations and constructive solid geometry 
 operators in the compact \blob structure. Our framework also provides a software architecture for 
 physically-based animation of rigid and deformable models.
 
 \item An algorithm for interactive polygonization of implicit surfaces on multi-core architectures with 
 SIMD instructions. (Peer reviewed contribution \cite{Shirazian2012}) 
 
 \item An optimized GPU-assisted algorithm for high-performance polygonization of implicit surfaces 
 on many-core architectures. As opposed to related work in this domain which is based on static implicit
 functions or constant range data, our rendering method applies to dynamic, data-driven \blob scene graph
 for complex implicit models. 
  
 \item A high-performance algorithm for cutting rigid and deformable tissues interactively. 
 
 \item Smooth cutting of complex volumetric meshes to avoid producing the jagged lines without the 
 need for a post-processing step.

 \item A novel mesh data-structure suitable for storing dynamic meshes on the GPU to support real-time 
 modifications during cutting
 
\item Real-time Craniotomy simulation for neurosurgery and biopsy simulations.

\end{itemize}


\section{Overview}
In the next chapter we start by providing background material on implicit modelling technique, the 
\blob scene-graph and the concept of sketch-based, incremental modelling.
We continue by reviewing the related work in surface extraction algorithms from volumetric models and
mesh connectivity modifications. 

Chapter \ref{chapter:cpuPoly} presents our rendering framework to visualize complex \blob models using 
multi-core architectures. Building on the outcomes of chapter \ref{chapter:cpuPoly}, the improved results 
are reviewed in chapter \ref{chapter:GPUDiscretization}. 

In Chapter \ref{chapter:Cutting} we present one of the main contributions of this thesis which is the high 
performance tissue cutting. After a brief overview of the related work we present our novel technique 
in cutting complex soft tissues interactively. Chapter \ref{chapter:evaluation} showcases a skull craniotomy 
simulation scenario and provides comments on the operation itself and the achieved results.

In Chapter \ref{chapter:conclusion} we provide a summary of the results in the previous chapters and 
review the limitations of the current system and some discussions on the future work in this research topic.















