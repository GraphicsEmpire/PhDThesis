\startchapter{GPU discretization}
\label{chapter:GPUDiscretization}
In the previous chapter we presented an algorithm for fast triangle mesh extraction from \blob scene graphs. The output of that 
algorithm is a list of vertices with their associated attributes such as position, color and normal and a list of triangles that
defines the connectivity of the surface mesh. While the surface mesh is used for rasterization it can also facilitate  
collision detection and contact modeling algorithms as we will see in the following chapters.  In this chapter we propose an improved 
version of that algorithm that can take advantage of the processing capabilities of the GPU and provides real-time \blob rendering performance. 



One of the primary objectives in our modeling framework was to be able to shift from sketching mode into live interactive animation mode
which can enable further enhancements; such as assignment of elastic parameters to the model in an incremental process, e.g. the
modeler can interactively make certain parts of the object stiffer and check out the results by haptically interacting with it or
visually examining the contacts between colliding bodies and with avatar. To make the modelled object amenable to this type of analysis, it has 
to be decomposed into simple shapes called \textit{elements}, which commonly are tetrahedra \cite{Labelle2007}. 
The success of the finite element method depends on the shapes of these tetrahedra, i.e. large dihedral angles cause large interpolation 
and discretization errors and will degrade the accuracy of the numerical simulation, small dihedral angles on the other hand will 
lead to incorrect stiffness matrices associated with the finite element method \cite{Shewchuk}. 


In order to address the tetrahedra extraction stage of the process we propose a GPU accelerated volume tetrahedralization algorithm based on the work of
Labelle \etal \cite{Labelle2007} that produces a smooth tetrahedral mesh. As opposed to their work our algorithm is scalable to number of processors 
available on the graphics processing units and delivers real-time results to support our modeling system.

The contributions in this chapter are as following:

\begin{itemize}
 \item A GPU accelerated polygonization algorithm for rendering complex \blob models. The algorithm scales with the number of stream cores available
 on modern GPU and can be easily extended to support various geometric primitives.
 
 \item A Tetrahedralization algorithm that leverages GPU for high performance volume discretization from \blob models. The output tetrahedral mesh is
 readily available for further physically-based simulations.
\end{itemize}


\section{Related Work}
GPU accelerated rendering techniques has been the topic of interest for the graphics community in the past two decades.
Several GPU-accelerated algorithms have been proposed for fast triangulation and rendering of iso-surfaces that are defined by
volume data-sets, algebraic surfaces and radial-basis functions. In this section we will review the most related works.

Chochl{\'i}k \etal proposed a GPU accelerated polygonization algorithm for dynamically changing implicit surfaces \cite{chochlik2012gpu}. 
Their method is based on the marching tetrahedra (MT) algorithm. The model is partitioned into cubic cells first and then each cell is 
subdivided into 6 tetrahedra to be further processed using the GPU geometry shading stage. The vertices are marked inside if their 
associated field is above zero and outside otherwise. A configuration index is computed per each tetrahedra based on the inside/outside 
vertices. The triangle mesh is produced which is shaded using the fragment shader stage. No further analysis has been made on the performance 
of their algorithm and the input models are limited to time varying simple algebraic surfaces. The used a linear interpolation root finding
method which produces low quality output. 

Buatois \etal proposed a GPU accelerated isosurface extraction method based on MT, similar to Chochl{\'i}k \etal \cite{Buatois2006}. The 
texture memory to transfer the position and field values of the grid vertices. They presented an analysis of the performance of their algorithm
using a fluid simulation volume data-set. They reported that excessive texture fetches can be a bottleneck in the performance of their method.


Tatarchuk \etal proposed a GPU iso-surface extraction algorithm which is a hybrid of marching cubes and marching tetrahedra \cite{Tatarchuk2007}. 
They start by voxelizing an implicit grid into discrete cube cells and then convert that to a tetrahedral representation. They implemented an MT 
algorithm on geometry shader stage of DirectX10 API. For root finding method they fitted a parabola along the intersecting edges and evaluated a 
quadratic equation which produces a better approximation. They tested their system using the visible human volume data-set. Their method
is limited to static volume datasets and is not usable in a data-driven setting where the topology of the underlying model changes. 

With modern hardware and fast GPUs, ray tracing of implicit surfaces is the subject of much research. Knoll \etal \cite{Knoll2009} presented 
CPU and GPU algorithms that can achieve interactive visualization for common algebraic surfaces. The surfaces used in by Knoll \etal are not 
arbitrary implicit models but surfaces generated using traditional kernels or functions. Similar surfaces are presented by Singh and Narayanan
for real-time ray tracing of implicit surfaces using the GPU \cite{singh2010real}. 


Kipfer and Westermann \cite{Kipfer2005} accelerated rendering of implicit surfaces by avoiding redundant computation of edge surface intersections. 
Our method also employs this feature to reduce the overhead. They also use features of the GPU to reformulate the iso-surface identification and reduces 
numerical computations and memory access operations. They used a span-space data-structure to avoid processing non surface intersecting elements.

Kanai \etal \cite{Kanai2006a} proposed a rendering method for sparse low-degree implicit surfaces (SLIM) using ray casting approach. The ray and IS intersection
test has been carried out on the fragment processing stage. The employed level of detail rendering and view frustum culling to speedup the rendering. 
The coefficients for the IS are passed in using textures. They reported high quality and interactive rates for several models. The large number of processed
fragments is the bottleneck in this process and models with lower number of nodes could be rendered slower than more complex models that cover less fragments.

%Another requirement for the tetrahedral mesh generation is that the triangular surface mesh should be the boundary surface of the tetrahedral mesh. This is needed since 
%after applying the displacements to the volumetric tetrahedral elements the surface vertices should be displaced as well.
\section{Many-cores architectures}
Shrinking of CMOS circuitry has allowed distances between transistors to scale fairly consistently for an extended period of time. The shrinking 
of distances and reduction in size of capacitors allowed hardware architects to clock circuits at a higher rate. This led to Gordon Moore's
famous self-fullfilling prophecy about transistor density and its misinterpretations into the realm of execution frequency and overall performance.

Certainly, increasing frequency allowed the performance of nonparallel code to increase consistently during that time, such that it became an 
expectation for software developers until the early 21st century. During the past decade, it has become obvious the continued scaling of clock 
frequencies of CPUs is not practical, largely due to power and heat dissipation constraints. The reason for this is that power consumption is 
dependent on frequency in a nonlinear manner \cite{gaster2012heterogeneous}. As a second problem, increasing clock frequency on-chip requires 
either increasing off-chip memory bandwidth to provide data fast enough to not stall the linear workload running through the processor or increasing
the amount of caching in the system. 

Due to those reasons the computational hardware available in high-performance workstations shifted from increasingly efficient but complex sequential 
computational units, to smaller units which are not faster than previous generations but the cores are duplicated to be able to execute more threads 
in parallel. This new trend in design can be found in modern CPUs and recent Graphics Processing Units (GPUs). The latest generation of GPUs contains 
hundreds of computation units (4096 in AMD Radeon 7990, 2496 in NVIDIA Tesla K20. For a brief discussion on how to count GPU cores refer to \cite{Fatahalian2008})

This radical architectural change has important consequences on the type of algorithms which are applicable in interactive simulations. In terms of 
programming, general purpose computations on GPUs initially required the use of graphics oriented libraries. The two major GPU vendors released general
programming APIs, CUDA \cite{Nickolls2008} and CTM \cite{peercy2006performance} which provide direct access to the underlying parallel processors of the
GPU, as well as full instruction sets, such as double precision computations and write operations at arbitrary locations. OpenCL which is a multi-vendor
standard was released in 2009, with a programming model very similar to CUDA \cite{gaster2012heterogeneous}. The presented algorithms in this chapter
have been implemented in OpenCL. 