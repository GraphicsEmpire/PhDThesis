\startchapter{Evaluation, Analysis and Comparisons}
\label{chapter:evaluation}
During the past decade a new thrust has appeared with the development of tissue modelling techniques 
which could make it possible to develop patient-specific simulations. Whenever the best surgical strategy 
is unclear or the patient presents a rare pathology such simulations could be beneficial. Other use-cases for such systems is in surgical skill training which is a long 
and tedious process of acquiring fine motor skills. In all those scenarios physically-based animation of 
tissues is challenging and has lots of room for improvement. 

Our methods presented in the previous chapters have been integrated into our surgical simulation system. 
The result is a comprehensive environment for tissue modelling and simulation with support for cutting. 
We present our results in the context of a skull craniotomy procedure.

Craniotomy is a surgical operation in which a part of skull, called a bone flap is temporarily removed to access 
the brain. The operation is performed on patients with brain lesions, traumatic brain injury 
(TBI) or for brain biopsy purposes. Some treatment procedures are done by deep implantation of brain 
stimulators; Parkinson disease, epilepsy and cerebellar tremor are examples of such cases that require 
a craniotomy for the implantation process. 

Small dime-sized craniotomies are called burr holes or keyhole craniotomies. In 
order to precisely control surgical and biopsy instruments through these small holes, the surgeons 
frequently use image-guided computer systems or endoscopes. Burr holes or keyhole craniotomies are 
used for minimally invasive procedures to:

\begin{itemize}
 \item insert a shunt into the ventricles to drain cerebrospinal fluid (hydrocephalus)
 \item insert a deep brain stimulator to treat Parkinson Disease
 \item insert an intracranial pressure (ICP) monitor
 \item remove a small sample of abnormal tissue (needle biopsy)
 \item drain a blood clot (stereotactic hematoma aspiration)
 \item insert an endoscope to remove small tumors and clip aneurysms
\end{itemize}

In the following sections we review the previous work in this domain and then describe our experimental 
setup . The chapter concludes with results and analysis.

\section{Previous work}
Abe \etal fabricated plastic skull models of seven individual patients by stereolithography from three-dimensional
data based on computed tomography (CT) bone images \cite{Abe1998}. Surgical approaches and areas of craniotomy were 
evaluated using the fabricated skull models. They reported a better understanding of anatomic relationships, preoperative
evaluation of the proposed procedure and improved educational experiences for the residents and medical staff as the benefits of 
their system. They also reported that the time and cost of making such models are the main disadvantages of using them.

Wong \etal loaded patient specific CT scans of cranial bone and CT angiography of intracranial circulation
into the Dextroscope workstation supplied by Volume Interactions Pte. Ltd \cite{Wong2007}. They showed various 
use-cases of the zoom, rotate, move and crop functions of the Dextroscope to visualize various angles of 
positioning the craniotomy. However, their system does not provide a physically-based simulation of the procedure. 

Stadie \etal performed a study on the effectiveness of virtual reality systems for placing the craniotomies 
in minimally invasive procedures \cite{Stadie2011}.  They used the Dextroscope and RadioDexter workstations 
supplied by Volume Interactions Pte. Ltd. to visualize and annotate the 3D VR models. 
Those systems are also used to measure the curvilinear distances of the proposed craniotomy centers on the 
patients skull model but they can not perform a cutting procedure on the input VR models.

\section{Architectural constraints}
We are able to perform interactive cuts on a model with more than 60,000 cells. The GPU accelerated cutting algorithm 
presented in chapter \ref{chapter:Cutting} requires a modern GPU with OpenCL support. The system we used for this
experiment is equipped with an Nvidia Geforce GTX760 with 2GB of video memory and 1152 CUDA cores. The CPU is an 
Intel Core i7-4770K with 256 KB, 1MB and 8MB of L1 to L3 cache. This processor has 4 cores and up to 8 threads can run in parallel. 
Our system is also equipped with 16 GB of main memory. 

\section{Experiments}
In this section we review the experiments that are performed to evaluate the 
performance and quality of our cutting algorithm in the context of the Craniotomy procedure. 
In the first experiment we use an eggshell model that helps us to study the 
quality of the cut surfaces and the amount of time it takes to apply mesh modifications in the associated 
data-structures. Fewer finite element cells in the eggshell model helps us to isolate the mesh quality 
issues in the vicinity of the cut surfaces. 

Later we use the human skull model and drill multiple holes at different locations in the bone tissue.  
The location of the drills are chosen according to the recorded videos of various brain biopsy 
procedures. Beside the experiments with the drilling tool, the scalpel tool is also used for the visualization 
of the internal mesh cross-sections, which can be helpful in the study of scanned volume data-sets from 
arbitrary viewing angles. 

Our system also allows controlled cutting along the major axes. The main benefit of 
this feature is in studying the effects of vibrations and other hand movements when 
cutting models interactively, i.e. by comparing the axis-locked, controlled cutting to 
the same scenario with free-hand movements, one can investigate the effect of 
small vibrations along the cut trajectory and how it can lead to the generation 
of small and wedge-shaped elements in the volume mesh. 

\subsection{The Eggshell Model}
\label{sec:eggshell}
Before drilling an actual skull mesh with many tetrahedral elements we tested our system on a spherial model similar to 
an eggshell. This model is composed of 840 nodes and 2280 tetrahedral cells. The outer surface is composed of two layers 
of tetrahedral elements only (see figure \ref{fig:eggshell01}). To create this model implicitly a smaller sphere is subtracted 
from a larger one. 

\begin{figure}[H]
  \centering
  % the following command controls the width of the embedded PS file
  % (relative to the width of the current column)
  \includegraphics[width=0.7\linewidth]{figures/evaluation/eggshell01.png}
  \caption{\label{fig:eggshell01}
  {Eggshell model before being drilled by our cutting tool.}
}
\end{figure}

After the drilling operation, only 120 new elements are added to the mesh for a total of 2400 elements. The entire process is completed 
interactively. The small disjoint parts fall down on the ground due to gravity. Each disjoint part becomes an independent
deformable model in our system and subject to forces and deformations. Figure \ref{fig:eggshell02} left, shows the mesh after being 
drilled for the bare hole. In order to perform a stress test on the model 6 
holes are drilled in various locations of the eggshell model (see figure \ref{fig:eggshell02}, right).


\begin{figure}[H]
  \centering
  % the following command controls the width of the embedded PS file
  % (relative to the width of the current column)
  \includegraphics[width=0.4\linewidth]{figures/evaluation/eggshell02.png}
  \includegraphics[width=0.4\linewidth]{figures/evaluation/eggshell05.png}
  \caption{\label{fig:eggshell02}
  {Eggshell model after being drilled by our cutting tool. Left: The first drill, Right: 
  After drilling 6 holes to the model.}
}
\end{figure}

After completing this experiment successfully our algorithm showed to be robust enough to handle 
more complex meshes. In the next stage a real data-set of skull tissue is used for the craniotomy operation.

\subsection{Craniotomy}
\label{sec:craniotomy}
Fang \etal published a tetrahedral mesh data-set of the brain tissues \cite{fang2010mesh}. The MRI scanned data is 
segmented into the following four regions:

\begin{enumerate}
 \item Skull and scalp
 \item Cerebro-spinal fluid (CSF)
 \item Gray-matter
 \item White-matter
\end{enumerate}

The high-resolution version of these segments are not published at the time of this writing so we used the lower resolution 
which has enough details of the organ for our simulation scenarios. The data-set is converted from its original format 
(MATLAB mat file) to our volumetric mesh format. Table \ref{table:brainmesh} shows number of nodes, edges, faces and tetrahedral cells 
per each segment after the conversion process:

\begin{table}[H]
\begin{center}
\caption{\label{table:brainmesh}{Segmented brain data-set statistics.}}
  \begin{tabular}{ | l | c | c | c | c |}
    \hline    
     & skull & csf & gray matter & white matter \\ \hline \hline    
    Nodes & 14739 & 37136 & 50741 & 23737  \\ \hline
    Edges & 89681 & 181593 & 268300 & 126441 \\ \hline
    Faces & 141498 & 251823 & 384989 & 184536 \\ \hline
    Cells & 66554 & 107460 & 167528 & 81833 \\ \hline
    \hline
  \end{tabular}
\end{center}
\end{table}

Due to its stiff material properties the skull tissue is modelled as a rigid material in our simulation system. 
Figure \ref{fig:craniotomy01} shows the skull mesh in its initial position.

\begin{figure}[H]
  \centering
  % the following command controls the width of the embedded PS file
  % (relative to the width of the current column)
  \includegraphics[width=0.7\linewidth]{figures/evaluation/craniotomy01.png}
  \caption{\label{fig:craniotomy01}
  {The scene setup for the craniotomy operation.}
}
\end{figure}


The cutting tool in this scenario is a tube-shaped device which can drill into the skull tissue and separate the bone matter.
In our system, the tool is defined as a curve approximated by $N$ line segments. The tool movement is tracked in the space 
and the system checks for collisions between the tool and the model continuously. Figure \ref{fig:craniotomytube} shows the 
polygonal shape of the cutting tool while in contact with the skull tissue.

\begin{figure}[H]
  \centering
  % the following command controls the width of the embedded PS file
  % (relative to the width of the current column)
  \includegraphics[width=0.6\linewidth]{figures/evaluation/craniotomytube.png}
  \caption{\label{fig:craniotomytube}
  {Cutting tool is defined as a tube with a base composed of a curve approximated with $N$ line segments. 
  Collisions between the tool and the tissue are monitored constantly.}
}
\end{figure}

When in contact with the skull tissue the intersection of the side wall of the tool is computed against the edges of the 
skull model. With $N$ line segments the cutting tool has $N$ quadrilateral faces on its side wall, the edge intersection 
test is performed in our system by calling the kernel function given in algorithm \ref{alg:edgeIntersections} once per each quad.
After each call the hash-table storing the cut-edges is filled with the new cuts. 

The cutting configurations presented in section \ref{sec:cutconfigs} are extracted based on one intersection per edge, therefore
it's not possible to cut a given edge more than once. In our system we only accept the first cut-edge and this did not produce any 
visual defects. Perhaps a more robust implementation would be to approximate the cutting tool curve based on the size of the longest 
edge in the mesh. After the cutting is made the mesh is separated and each of the disconnected parts is converted into a separate node
in our scene-graph structure. This operation results in correct detection of the self-collisions in the subsequent frames of the simulation.

The internal gray, white and CSF matters are also included in this simulation. The drilling operation only affects the skull tissue and
in fact it does not pass the Dura layer. This condition is a requirement for the successful completion of this procedure. 

Figure \ref{fig:crosssection} shows the cross section view of the brain. The skull is cut with a scalpel tool 
to show the internal tissues which are drawn in blue for better visibility. 
\begin{figure}[H]
  \centering
  % the following command controls the width of the embedded PS file
  % (relative to the width of the current column)
  \includegraphics[width=0.5\linewidth]{figures/evaluation/crosssection.png}
  \caption{\label{fig:crosssection}
  {Cross section view of the brain layers. The skull shown in pink is cut using a scalpel avatar to show 
   all the other layers depicted in blue.}
}
\end{figure}


\begin{figure}[H]
  \centering
  \includegraphics[width=0.4\linewidth]{figures/evaluation/craniotomy06.png}
  \includegraphics[width=0.4\linewidth]{figures/evaluation/craniotomy07.png}
  \includegraphics[width=0.4\linewidth]{figures/evaluation/craniotomy09.png}
  \includegraphics[width=0.4\linewidth]{figures/evaluation/craniotomy10.png}
  \caption{\label{fig:craniotomy}
  {Simulation of the craniotomy operation using our surgical simulation framework with support for interactive cutting.}
}
\end{figure}

Figure \ref{fig:craniotomy} shows the operation (before, during and after drilling the first and the second holes). 
During the drilling process 845 and 940 tetrahedral cells are being cut in the vicinity of the drilling tool 
for the first and second holes, respectively. The interaction of the drilling tool and the skull 
was interactive at all times, supporting at least 60 frames per second.