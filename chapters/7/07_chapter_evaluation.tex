\startchapter{Evaluation, Analysis and Comparisons}
\label{chapter:evaluation}
Our methods presented in the previous chapters have been integrated into our surgical simulation system. 
In this system, the methods have been combined with physical soft-tissue simulation and collision handling of deformable
models for an immersive surgical scenario. We present our results in the context of two operations.

The first operation is craniotomy which is a surgical operation in which a bone flap is temporarily removed from the skull to access the brain.
Craniotomies are often a critical operation performed on patients suffering from brain lesions or traumatic brain injury (TBI), and can also allow 
doctors to surgically implant deep brain stimulators for the treatment of Parkinson's disease, epilepsy and cerebellar tremor. The procedure is 
also widely used in neuroscience for extracellular recording, brain imaging, and for neurological manipulations such as electrical stimulation and 
chemical titration.

The second operation is brain biopsy which is the removal of a small piece of brain tissue for the diagnosis of abnormalities of the brain. It is used 
to diagnose Alzheimer's disease, tumors, infection, inflammation, and other brain disorders. By examining the tissue sample under a microscope, the 
biopsy sample provides doctors with the information necessary to guide diagnosis and treatment.