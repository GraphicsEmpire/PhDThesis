\startchapter{Real-time Cutting}
\label{chapter:Cutting}
One of the main objectives of virtual reality based surgical simulation systems is the removal of pathologic tissue 
\cite{Steinemann, Nienhuys2001a}. Cutting imposes many challenges in the development of a robust, interactive surgery 
simulation, not only because of the nonlinear geometric and material behavior exhibited by soft tissue but also due to the
complexity of introducing the cutting-induced discontinuity. In most publications, the progressive surgical cutting is modelled
by conventional finite element (FE) method, in which the high computational cost and error accumulation due to remeshing constrain 
the computational efficiency and accuracy. 

We developed our new cutting approach in the context of brain biopsy simulation. When an abnormality of the brain is suspected, 
Stereotactic (probing in three dimensions) brain needle biopsy is performed and guided precisely by a computer system to avoid 
serious complications. A small hole is drilled into the skull, and a needle is inserted into the brain tissue guided by computer-assisted 
imaging techniques (CT or MRI scans). Due to this, the actual cutting process can not be seen by the surgeon. For this reason,
non-progressive cutting, where a tetrahedral element is decomposed only once is has been completely traversed, is a reasonable
approximation for our application area, and so we define the cut only once the instrument has traversed the pathology. 
Moreover, there is little, if at all, resistance to the cut tool movement through the tissue. Therefore, in the current stage, we do 
not model any interaction of the cutting tool with the deformable object during a cut. 

The deformable tissues in our framework are represented by tetrahedral meshes and simulated using nonlinear finite element methods. 

\section{Related Work}
A number of approaches has been proposed by the computer graphics community to enable cutting deformable models. 
Except for a few methods most of them used tetrahedral meshes for the volumetric mesh representation. 
Bielser \etal performed an adaptive refinement of the tetrahedral elements cut by a virtual scalpel \cite{Bielser1999}.

Mor \etal tried to reduce the number of sub-elements created while cutting tetrahedral meshes \cite{Mor2000}.
One of the major issues in cutting tetrahedral meshes is the creation of ill-shaped elements which can adversely affect the
performance and stability of the system solver. Some works attempted to avoid such elements via mesh alignment techniques 
\cite{Nienhuys2001a, Steinemann2006}. Other methods tried to solve the issue by removing the ill-shaped elements completely.




Jin \etal proposed a meshless total Lagrangian adaptive dynamic relaxation cutting algorithm to predict the steady-state 
responses of soft tissue at any stage of surgical cutting in 3D \cite{Jin2013}. A cloud of points is used for discretization
and approximation of the deformation field within the continuum without generation of finite element meshes. They didn't report 
any performance measurements and the quality of the cuts could not be verified with the simple truth cube model they reported in their paper.


Wu \etal \cite{Wu2011} proposed an algorithm for 3D mesh cutting using a combination of the adaptive octree refinement with iterative composite
element hierarchy to enable simulating high-resolution cuts with a small number of degrees of freedom (DOFs). 
They used the dual contouring method \cite{Ju2002} to keep the sharp creases along the cut. Due to the high computational cost and naive implementation 
their method is not scalable and has yet to become an interactive cutting approach.



%Most of these challenges are due to the contradictory requirements of speed versus the fidelity of the cutting surfaces. 
%%In our experience the following major points required extra effort and validation:

